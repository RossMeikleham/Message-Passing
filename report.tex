\documentclass[fleqn]{report} 
\usepackage{listings}
\usepackage[margin=1.3in]{geometry}
\begin{document}

\title{Advanced Operating Systems(M) Exercise 3}
\author{Ross Meikleham\\ 1107023m}
\maketitle

\section*{Choice of Language:}

The message passing examples with Java + Akka that that I
found were extremely verbose and boilerplate heavy.\\
A "Hello World" example I found involved having to import multiple Akka classes, 
extending Actor classes, and overriding multiple methods. \\
The process of creating actors, and message sending also seems very verbose; for example to create
an actor in the "Hello World" example it involves the following statement:
\vspace{2mm}
\begin{lstlisting}[basicstyle=\ttfamily\scriptsize, language=Java]
final ActorRef greeter = getContext().actorOf(Props.create(Greeter.class), "greeter");
\end{lstlisting}
\vspace{2mm}
Scala with Akka seemed much nicer, it still involves importing Akka classes and 
extending classes to create actors, but examples I found were
far more simpler and less boilerplate heavy than Java + Akka versions.\\

I decided to choose Erlang due to the dynamic nature of the language which makes
it much less verbose than the other two languages. Having message passing built in as 
part of the language is helpful as it removes the need to download and install
an extra library.
The purely functional nature of the language helps with the "safety" of message passing,
such as enforcing that messages not being altered after they are sent due to immutability. 

The disadvantages of this choice I felt at the time were that I didn't know the language, 
and I don't have nearly as much experience with functional programming languages than I do 
with imperative/OOP languages. Also the syntax of the Erlang language is unlike any other 
language that I've ever encountered. These disadvantages could mean that it could
take longer to implement the system than if I used a language in a paradigm 
I was more familiar with. Also Erlang doesn't seem to be a very popular language meaning
that documentation, tutorials, and discussion on problems with the language on the web
could possibly be lacking compared to Java and Scala.

\section*{Installation:}

The installation process for erlang on my system was extremely simple.
I literally had to enter one command and my package manager installed it with
no problems.\\ 

Using the LearnYouSomeErlang tutorials made it trivial to get some example
programs running and learning the basic language features/syntax. 
Although my suspicions were confirmed in that general tutorials and documentation seemed 
very lacking compared to any other programming language I've ever used. 
If it wern't for the LYSE tutorials then I think I would have found it a much more
frustrating experience learning the language.\\  

The language itself was rather strange in regards to syntax, 
I've never encountered a language which uses
commas to seperate statements and a full-stop to mark the last one. A bit more confusing was
if statements/case statements when having to use a semi colon to mark the end of those cases
except the last case. However I got used to the syntax pretty quickly.
It helped that I installed a plugin for my text editor a while ago which highlights warnings
and errors with source files, and was surprised to see it worked with Erlang, and it was
a huge help in pointing out the numerous simple syntax errors as I was making them.

\section*{Design:}

I designed the system such that each actor is represented by a function. Each actor
except the clock actor waits to recieve a message, works out the type of message
using pattern matching, and then sends another message to another actor. Using 
recursion each actor can wait for the next message after performing the action
required from a previous message.\\

To start the system, a function is called which spawns all actors except the clock actor
and passes the appropriate Process Ids to each actor so they can send messages
to the correct processes. The clock actor is then called with a given time to run for.\\

The clock actor, waits for 1 second, then sends a message to both the farenheit actor,
and celcius actors to read temperatures. It then recursively calls itself lowering
the time by 1 second, and after the time becomes 0 it then doesn't call itself
again, ending the recursive loop. Then the system exits the rest of the processes
still running for the 4 remaining actors.\\

Some unusual features of my design is that the system takes a time value in milliseconds
to run for, so I had control over starting and stopping the system. Also the sensors
themselves take a predefined list of temperatures (so I was able to test the conversions)
and then if the lists run out then just generate random numbers to convert.\\

\section*{Overall Opinion:}

One thing I like about the system is that any actor can be replaced without
affecting the others. For example the temperature sensors could be replaced with
functions which actually calculate the current temperature and none of the other
actors would need to be changed at all.\\

Although one thing I dislike and am skeptical about is error handling, as the
erlang philosophy is to let it crash and let a higher process deal
with the error. Depending on the implementation of an actor there could be
different kinds of errors which need to be accounted for.\\

I don't thing the erlang message passing system is suitable for non-expert
programmers to use. The syntax and general functional paradigm 
used in the language would take too long for non-expert programmers to get 
used to. Most non-expert programmers would be used to imperative/OOP languages
such as Java, and even Scala can be used in an OOP way. So I think non-expert
programmers would be better off using a more familiar language such
as Scala or Java and learning the Akka library so they only have
to focus on learning the basics of message passing and not an entire
programming language. \\

Also erlang seems like a very specialised language that is
suited for very specific problems. Using it purely for
the fact it has message passing as a language feature 
seems overkill compared to using a more "general purpose" 
language and learning a message passing library.

I think message passing itself seems like a very good idea for specific 
problems, it was very easy to use due the fact it takes away having
to manually manage threads and locks, and helps avoid race conditions.
  
\newpage
\section*{Code Listing:}

\lstinputlisting[basicstyle=\ttfamily\scriptsize, language=erlang]{temp_converter.erl}


\end{document}
