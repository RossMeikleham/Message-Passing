\documentclass[fleqn]{report} 
\usepackage{listings}
\begin{document}

Choice of Language:

I've used both Scala and Java before, so I decided to try out Erlang. The installation process was
trivial for my system, as my package manager took care of it.

Running programs was quite straight forward, it works the same way as Java, having to compile then run although
I had to find out that you needed a certain flag when running the erlang shell to execute the compiled 
erlang source file, as otherwise it just brought me straight to the shell.

However the error messages were
very unhelpful. Again it took me a while to realise the "expected" way to run erlang programs is
to open a shell session, compile and run from the shell. I was then getting helpful error
messages running my programs this way.

At the point I'd like to point out the general searching for how to actually find out
these methods was very frustrating compared to any other language I've ever used. 
General tutorials were very sparse, and sites such as Stackoverflow didn't have
any questions/answers on very basic topics. I'm not sure if this is down to
the language not being that popular, but at points it was almost changing my
mind to work with Scala. 

The language itself was rather strange in regards to syntax, never encountered a language which uses
commas to seperate statements and a full-stop to mark the end. A bbit more confusing was
if statements/case statements when having to use a semi colon to mark the end of those cases
and the last case is not marked with anything. However I got used to the syntax pretty quickly.
It helped that I installed a plugin for my text editor a while ago which highlights warnings
and errors with source files, and was surprised to see it was working with Erlang, it helped
point out simple syntax errors as I was making them.

Message Passing

Once I got over the initial hurdle of running general programs, and learning the syntax of the language,
as well as discovering documentation on the libraries available I decided to start learning about
message passing. I was quite surpised the documentation and tutorials I could find on this
topic with Erlang was really good and I built a simple example really quickly.

The message passing in Erlang was pretty good.


\lstinputlisting[basicstyle=\ttfamily\scriptsize, language=erlang]{temp_converter.erl}


\end{document}
